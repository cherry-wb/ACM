\documentclass[a4paper]{article}
\usepackage{geometry}
\usepackage{fontspec}
\usepackage{listings}
\lstset{breaklines,numbers=left,frame=shadowbox,extendedchars=false}
\geometry{left=3cm,right=3cm,top=2.5cm,bottom=2.5cm}
\setmainfont{WenQuanYi Micro Hei}
\begin{document}  
\title{ACM/ICPC TEMPLATE}  
\author{NKU -> HOT}    
% generates the title
\maketitle
% insert the table of contents
\tableofcontents
\section{Introduction}
NKU -> HOT ACM-ICPC template \\
Thanks all past teammates and contributers! 
\section{Utility}
\subsection{Java Template}
\lstinputlisting{javatemp.java}
\subsection{Java Multithread}
BECAREFUL: CALL START FOR EACH THREAD!
\lstinputlisting{javaThread.java}
\subsection{Binary Search}
MAKE SURE check(x) is monotone in [L,R) \\
MAKE SURE check(L) == TRUE AND check(R) == FALSE FIRST!
\lstinputlisting{binarysearch.cpp}
\section{Graph Theroy}
\subsection{Prim - O($N^2$)}
\lstinputlisting{primNaive.cpp}
\subsection{Prim- O(MlogN)}
\lstinputlisting{primPQ.cpp}
\subsection{Kruskal -O(MlogM)}
\lstinputlisting{kruskal.cpp}
\subsection{Dijkstra - O($N^2$)}
\lstinputlisting{dijNaive.cpp}
\subsection{Dijkstra - O(MlogN)}
\lstinputlisting{dijPQ.cpp}
\subsection{Dijkstra with heap}
\lstinputlisting{dijHeap.cpp}
\subsection{Bellman-Ford}
\lstinputlisting{bellmanFord.cpp}
\subsection{Shortest Path Faster Algorithm}
\lstinputlisting{SPFA.cpp}
\subsection{Network Flow - ISAP[NON-ORIGINAL]}
\lstinputlisting{isap.cpp}
\subsection{Bipartite Graph Matching}
\lstinputlisting{matching.cpp}
\subsection{Minimun Cost Flow [TO BE TESTED!]}
\lstinputlisting{minimumcostflow.cpp}
\subsection{Kuhn-Munkras [NON-ORIGINAL]}
refined from http://blog.sina.com.cn/s/blog\_6ec5c2d00100vt8d.html
\lstinputlisting{km.cpp}
\subsection{Cut Vetrix and Edge}
\lstinputlisting{cut.cpp}
\subsection{Strongly Connected Components}
\lstinputlisting{scc.cpp}
\section{String Algorithm}
\subsection{ELF Hash}
\lstinputlisting{elfhash.cpp}
\subsection {Aho-Corasick Automation}
\lstinputlisting{ac.cpp}
\section{Data Struct}
\subsection {Binary Indexed Tree}
BECAREFUL WHILE I == 0 !!!
\lstinputlisting{bit.cpp}
\subsection {Inversion}
\lstinputlisting{inversion.cpp}
\subsection {BigInt Multiply with FFT}
\lstinputlisting{mul.cpp}
\section{Computational Geometry}
\subsection{Constants}
\begin{lstlisting}
  const double eps=1e-12;
  const double pi=acos(-1.0f);
  const double INF=1.0/0.0f;
  const double NEGINF=-1.0/0.0f;
\end{lstlisting}
\subsection{Compare Function}
\begin{lstlisting}
  inline bool equ0(const double& x){return fabs(x)<eps;}
  inline bool equ(const double& x,const double& y){return fabs(x-y)<eps;}
  inline bool ls(const double& x,const double& y){return x+eps<y;}
  inline bool gr(const double& x,const double& y){return x-eps>y;}
  inline bool greq(const double& x,const double& y){return x+eps>=y;}
  inline bool lseq(const double& x,const double& y){return x-eps<=y;}
\end{lstlisting}
\subsection{3D Point and Vector}
\begin{lstlisting}
  struct Point{
    double x,y,z;
    explicit Point(const double& a=0,const double& b=0,const double& c=0):x(a),y(b),z(c){}
    Point(const Point& p):x(p.x),y(p.y),z(p.z){}
    Point operator-() const{return Point(-x,-y,-z);}
    Point operator-(const Point& a) const{return Point(x-a.x,y-a.y,z-a.z);}
    Point operator+(const Point& a) const{return Point(x+a.x,y+a.y,z+a.z);}
    Point operator*(const double& s) const{return Point(s*x,s*y,s*z);}
    double sqlen() const{return x*x+y*y+z*z;}
    double len() const{return sqrt(sqlen());}
    void norm(){double l=len();x/=l;y/=l;z/=l;}
  };
  typedef Point Vec;
\end{lstlisting}
\subsection{3D Utility Functions}
\subsubsection{3D dot and 3D vector cross}
\begin{lstlisting}
  inline double dot(const Vec& u,const Vec& v) 
  {return u.x*v.x + u.y*v.y + u.z*v.z;}
  inline Vec cross(const Vec& u,const Vec& v){
    return Vec(u.y*v.z-u.z*v.y,u.z*v.x-u.x*v.z,u.x*v.y-u.y*v.x);
  }
\end{lstlisting}
\subsubsection{3D line - line min distance}
POJ 2852
\begin{lstlisting}
  inline bool line_mindist(const Point& p0,const Vec& d0,
  const Point& p1,const Vec& d1,
  double* t,double* k){
    double a=dot(d0,d0),b=-dot(d0,d1),c=-dot(d1,d1),d=dot(d0,p1-p0),
    e=dot(d1,p1-p0);
    double det=a*c+b*b;
    if(equ0(det)) return false;
    *t=(d*c-b*e)/det;
    *k=(a*e+b*d)/det;
    return true;
  }
\end{lstlisting}
\subsubsection{Sphere coord}
\begin{lstlisting}
  inline double deg2grad(const double& d){return d*pi/180.0;}
  inline Vec sphere_coord(const double& ele,const double& az){
    double x,y,z,r;
    z=sin(ele);
    r=cos(ele);x=r*cos(az);y=r*sin(az);
    return Vec(x,y,z);
  }
\end{lstlisting}
\subsection{2D Point and Vector}
\begin{lstlisting}
  struct Point{
    double x,y;
    explicit Point(const double& a=0,const double& b=0):x(a),y(b){}
    Point(const Point& p):x(p.x),y(p.y){}
    Point& operator=(const Point& p){x=p.x;y=p.y;return *this;}
    Point operator-() const{return Point(-x,-y);}
    Point operator-(const Point& a) const{return Point(x-a.x,y-a.y);}
    Point operator+(const Point& a) const{return Point(x+a.x,y+a.y);}
    Point operator*(const double& s) const{return Point(s*x,s*y);}
    double sqlen() const{return x*x+y*y;}
    double len() const{return sqrt(sqlen());}
    void norm(){double l=len();x/=l;y/=l;}
  };
  typedef Point Vec;
\end{lstlisting}
\subsection{2D Utility Functions}
\subsubsection{2D dot and 2D scalar cross}
\begin{lstlisting}
  inline double dot(const Vec& u,const Vec& v){return u.x*v.x + u.y*v.y;}
  inline double cross(const Vec& u,const Vec& v){return u.x*v.y - u.y*v.x;}
\end{lstlisting}
\subsubsection{2D triple point coline}
\begin{lstlisting}
  inline bool coline(const Point& a,const Point& b,const Point& c){
    Vec u=b-a,v=c-a;
    return equ0(cross(u,v));
  }
\end{lstlisting}
\subsubsection{2D segment - segment overlap}
\begin{lstlisting}
  inline bool seg_overlap(const Point& a1,const Point& a2,const Point& b1,const Point& b2){
    if(coline(a1,a2,b1) && coline(a1,a2,b2)){
      double k1,k2;
      Vec axis=a2-a1;
      Vec temp=b1-a1;
      if(equ0(axis.y)){
        k1=temp.x/axis.x;
        temp=b2-a1;
        k2=temp.x/axis.x;
      }
      else{
        k1=temp.y/axis.y;
        temp=b2-a1;
        k2=temp.y/axis.y;
      }
      if((gr(k1,1.0f) && gr(k2,1.0f)) || (ls(k1,0.0f) && ls(k2,0.0f))) return false;
      else return true;
    }
    else return false;
  }
\end{lstlisting}
\subsubsection{2D angle difference between two vectors}
\begin{lstlisting}
  //CCW for positive
  inline double angle_diff(const Vec& from,const Vec& to){
    Vec nfrom=from;nfrom.norm();
    Vec nto=to;nto.norm();
    double d=dot(nfrom,nto),c=cross(nfrom,nto),a=acos(d);
    if(ls(c,0.0f)) return 2.0f*pi-a;else return a;
  }
\end{lstlisting}
\subsubsection{2D point - line distance}
\begin{lstlisting}
  inline double dist2line(const Point& p,const Point& o,const Vec& d){
    Vec u=p-o,nd=d;nd.norm();
    double proj=dot(u,nd);
    return sqrt(u.sqlen()-proj*proj);
  }
\end{lstlisting}
\subsubsection{2D point - circle tangent}
POJ 1375
\begin{lstlisting}
  inline int point_circle_tan(const Point& p,const Point& C,const double& r,Point* tps){
    Vec v=C-p;double cdist=v.len();
    if(ls(cdist,r)) return 0;
    else if(equ(cdist,r)){tps[0]=p;return 1;}
    double l=sqrt(cdist*cdist-r*r);v.norm();v=v*l;
    double a=asin(r/cdist);
    Vec u=vec_rotate(v,a);tps[0]=p+u;
    u=vec_rotate(v,-a);tps[1]=p+u;
    return 2;
  };
\end{lstlisting}
\subsubsection{2D circle - circle tangent}
require circles no touch, no intersect \\
POJ 2416
\begin{lstlisting}
  struct Seg{
    Point p0,p1;
    Seg(const Point& m=Point(),const Point& n=Point()):p0(m),p1(n){
    }
  };

  inline void tan_2circle(const Point& C0,const double& R0,
  const Point& C1,const double& R1,
  Seg* segs){
    Point c0=C0,c1=C1;double r0=R0,r1=R1;
    if(gr(R1,R0)){swap(c0,c1);swap(r0,r1);}
    Vec v=c1-c0;double cdist=v.len();

    Point p=c0+v*(r0/(r1+r0));
    Point t[2];
    point_circle_tan(p,c0,r0,t);
    Vec u=p-t[0];Point w;
    w=t[0]+u*((r1+r0)/r0);
    segs[0]=Seg(t[0],w);
    u=p-t[1];w=t[1]+u*((r1+r0)/r0);
    segs[1]=Seg(t[1],w);

    if(equ(r0,r1)){
      u=v;u.norm();u=u*r0;
      Vec offset=vec_rotate(u,pi*0.5);
      segs[2]=Seg(c0+offset,c1+offset);
      offset=vec_rotate(u,-pi*0.5);
      segs[3]=Seg(c0+offset,c1+offset);
    }
    else{
      p=c0+v*(r0/(r0-r1));
      Point k[2];
      point_circle_tan(p,c1,r1,t);point_circle_tan(p,c0,r0,k);
      segs[2]=Seg(t[0],k[0]);
      segs[3]=Seg(t[1],k[1]);
    }
  }
\end{lstlisting}
\subsubsection{2D circle - polygon intersection area}
POJ 3675 \\
schindlerlee, thanks
\begin{lstlisting}
  //CCW
  inline double poly_area(const vector<Point>& p){
    Point prev=p[0],cur;
    double area=0;
    for(int i=1;i<p.size();++i){
      cur=p[i];
      area+=0.5 * cross(prev,cur);
      prev=cur;
    }
    return area;
  }

  double R;
  Point C;
  vector<Point> P;
  int main(){
    while(scanf("%lf",&R)==1){
    P.clear();
    int N;
    scanf("%d",&N);
    for(int i=0;i<N;++i){
      double x,y;scanf("%lf%lf",&x,&y);
      P.push_back(Point(x,y));
    }
    C=Point(0.0f,0.0f);
    P.push_back(P[0]);
    double area=0.0f;
    double t,k;
    for(int i=0;i<P.size()-1;++i){
      Point s0=P[i],s1=P[i+1];
      if(line_circle_intersect(s0,s1-s0,C,R,&t,&k)){
        Point p0=s0+(s1-s0)*t;
        Point p1=s0+(s1-s0)*k;
        if(lseq(k,0.0f) || greq(t,1.0f)){
          area+=0.5 * R*R * angle_diff(s0,s1);
        }
        else if(lseq(t,0.0f) && greq(k,1.0f)){
          area+=0.5 * cross(s0,s1);
        }
        else if(lseq(t,0.0f) && greq(k,0.0f) && lseq(k,1.0f)){
          area+=0.5 * cross(s0,p1);
          area+=0.5 * R*R * angle_diff(p1,s1);
        }
        else if(greq(k,1.0f) && greq(t,0.0f) && lseq(t,1.0f)){
          area+=0.5 * R*R * angle_diff(s0,p0);
          area+=0.5 * cross(p0,s1);
        }
        else{
          area+=0.5 * R*R * angle_diff(s0,p0);
          area+=0.5 * cross(p0,p1);
          area+=0.5 * R*R * angle_diff(p1,s1);
        }
      }
      else area+=0.5f * R*R * angle_diff(s0,s1);
    }
    printf("%.2f\n",fabs(area));
  }//while case
  return 0;
}
\end{lstlisting}
\subsubsection{2D vector normal}
return two normalized vectors perp to the v  \\
(CCW,CW)
\begin{lstlisting}
  inline pair<Vec,Vec> vec_normal(const Vec& v){
    Vec u=v;u.norm();
    Vec CCW=Vec(-u.y,u.x),CW=Vec(u.y,-u.x);
    return make_pair(CCW,CW);
  }
\end{lstlisting}
\subsubsection{2D vector rotation}
CCW for positive angle
\begin{lstlisting}
  inline Vec vec_rotate(const Vec& v,const double& a){
    double s=sin(a),c=cos(a);
    return Vec(c*v.x-s*v.y,s*v.x+c*v.y);
  }
  inline Vec vec_rotate_left(const Vec& v){return Vec(-v.y,v.x);}
  inline Vec vec_rotate_right(const Vec& v){return Vec(v.y,-v.x);}
\end{lstlisting}
\subsection{2D Intersection}
\subsubsection{ray - ray intersection}
\begin{lstlisting}
  inline bool ray_intersect(const Point& p,const Vec& u,
  const Point& q,const Vec& v,
  double* t,double* k){
    double a=u.x,b=-v.x,c=u.y,d=-v.y;
    double e=q.x-p.x,f=q.y-p.y;
    double det=a*d-b*c;
    if(equ0(det)) return false;
    else{
      *t=(e*d-b*f)/det;
      *k=(a*f-e*c)/det;
      return true;
    }
  }
\end{lstlisting}
\subsubsection{line - circle intersection}
\begin{lstlisting}
  inline bool line_circle_intersect(const Point& o,const Vec& d,
  const Point& C,const double& R,
  double* t,double* k){
    Vec delt=o-C;
    double a=d.sqlen(),b=2*dot(delt,d),c=delt.sqlen()-R*R;
    double x=b*b-4*a*c;
    if(ls(x,0.0f)) return false;
    x=sqrt(x);*t=(-b-x)/(2*a);*k=(-b+x)/(2*a);
    return true;
  }
\end{lstlisting}
\subsection{Computational Geometry Topics}
\subsubsection{2D convex hull}

O(NlogN) \\ 
monotone chain \\ 
vertex points only: modify the 'left' to strict, otherwise colinear points included. \\
CCW order \\
POJ 2187 \\
\begin{lstlisting}
  const int MAXP=50000;
  Point P[MAXP];
  int N;

  int H[MAXP],lower[MAXP];
  bool vis[MAXP];
  int top;
  int monotone_chain(){
    memset(vis,0,sizeof(vis));
    int utop=0,ltop=0;
    for(int i=N-1;i>=0;--i){
      if(utop<2) H[utop++]=i;
      else{
        while(utop>1 && cross(P[H[utop-1]]-P[H[utop-2]],P[i]-P[H[utop-1]]) <= 0) --utop;
        H[utop++]=i;
      }
    }
    for(int i=0;i<utop;++i) vis[H[i]]=true;
    for(int i=0;i<N;++i){
      if(ltop<2) lower[ltop++]=i;
      else{
        while(ltop>1 && cross(P[lower[ltop-1]]-P[lower[ltop-2]],P[i]-P[lower[ltop-1]]) <= 0)
        --ltop;
        lower[ltop++]=i;
      }
    }
    for(int i=0;i<ltop;++i) if(!vis[lower[i]]) H[utop++]=lower[i];
    return utop;
  }
\end{lstlisting}
\subsubsection{3D convec hull}
O(N2) \\
incremental method \\
POJ 3528 \\
schindlerlee, thanks.
\begin{lstlisting}
  //CCW
  struct Face{
    Point p0,p1,p2;
    Face(const Point& a=Point(),const Point& b=Point(),const Point& c=Point()):
    p0(a),p1(b),p2(c){
    }
  };
  inline Vec face_normal(const Face& f){
    Vec u=f.p1-f.p0;
    Vec v=f.p2-f.p0;
    Vec n=cross(u,v);
    n.norm();
    return n;
  }
  inline bool point_above_face(const Point& p,const Face& f){
    Vec u=p-f.p0;
    Vec n=face_normal(f);
    return gr(dot(u,n),0.0f);
  }


  int edge[MAXP][MAXP];
  int main(){
    int N;
    vector<Point> Points;
    vector<Face> CH;
    while(scanf("%d",&N)==1){
    for(int i=0;i<N;++i){
      double x,y,z;
      scanf("%lf%lf%lf",&x,&y,&z);
      Points.push_back(Point(x,y,z,i));
    }
    //special case

    CH.push_back(Face(Points[0],Points[1],Points[2]));
    CH.push_back(Face(Points[2],Points[1],Points[0]));
    
    for(int i=3;i<Points.size();++i){
      Point cur=Points[i];
      for(int j=0;j<CH.size();++j){
        Face f=CH[j];
        if(point_above_face(cur,f)){
          edge[f.p0.id][f.p1.id]=1;
          edge[f.p1.id][f.p2.id]=1;
          edge[f.p2.id][f.p0.id]=1;
        }
        else{
          edge[f.p0.id][f.p1.id]=-1;
          edge[f.p1.id][f.p2.id]=-1;
          edge[f.p2.id][f.p0.id]=-1;
        }
      }
      vector<Face> T;
      for(int j=0;j<CH.size();++j){
        Face f=CH[j];
        if(edge[f.p0.id][f.p1.id]==1){
          if(edge[f.p1.id][f.p0.id]==-1) T.push_back(Face(f.p0,f.p1,cur));
          if(edge[f.p2.id][f.p1.id]==-1) T.push_back(Face(f.p1,f.p2,cur));
          if(edge[f.p0.id][f.p2.id]==-1) T.push_back(Face(f.p2,f.p0,cur));
        }
        else{
          T.push_back(f);
        }
      }
      swap(CH,T);
    }
    double area=0.0f;
    for(int i=0;i<CH.size();++i){
      Face f=CH[i];
      Vec crs=cross(f.p1-f.p0,f.p2-f.p0);
      area+=0.5f * crs.len();
    }
    printf("%.3f\n",area);
  }
  return 0;
  }
\end{lstlisting}
\subsubsection{Max distance point pair}
O(NlogN) \\
applied on the convex hull CCW ordered points \\
rotating calipers \\
no explicit angle calculation \\
POJ 2187; TJU 2847
\begin{lstlisting}
inline int Next(int x){return (x+1)%top;}
int max_dist;
void rc(){
  int s;
  int p=0,q=Next(p);
  while(area(P[H[p]],P[H[Next(p)]],P[H[Next(q)]]) > 
  area(P[H[p]],P[H[Next(p)]],P[H[q]])){
    q=Next(q);
  }
  s=p;
  do{
    Vec d=P[H[p]]-P[H[q]];int temp=d.sqlen();
    max_dist=max(max_dist,temp);
    int np=p,nq=q;
    if(area(P[H[p]],P[H[Next(p)]],P[H[Next(q)]]) > 
    area(P[H[p]],P[H[Next(p)]],P[H[q]])) nq=Next(q);
    if(area(P[H[q]],P[H[Next(q)]],P[H[Next(p)]]) > 
    area(P[H[q]],P[H[Next(q)]],P[H[p]])) np=Next(p);
    if(nq!=q) q=Next(q);
    else p=Next(p);
  } while(q!=Next(s));
}
\end{lstlisting}
\subsubsection{closed point pair among two polygons}
O(NlogN) \\
rotating calipers \\
no explicit angle calculation \\
POJ 3608
utility functions \\
compare functor for monotone chain convec hull
\begin{lstlisting}
struct comp{
  bool operator()(const Point& a,const Point& b)const{
    if(equ(a.x,b.x)) return ls(a.y,b.y);else return ls(a.x,b.x);
  }
};
\end{lstlisting}
\begin{lstlisting}
inline double area(const Point& a,const Point& b,const Point& c){return cross(b-a,c-a);}

const int MAXP=10000;
int M,N;
Point P[MAXP],Q[MAXP];

double mindist;
\end{lstlisting}
utility functions to step to the next vertex. used in rotating calipers
\begin{lstlisting}
inline int nextp(int i){return (i+1)%M;}
inline int nextq(int i){return (i+1)%N;}
\end{lstlisting}
check if a point can perp project on a segment
\begin{lstlisting}
inline bool point_over_seg(const Point& p,const Point& a,const Point& b){
  Vec u=b-a;double l=u.len();
  u.norm();
  double proj=dot(u,p-a);
  double t=proj/l;
  return ls(t,1.0f) && gr(t,0.0f);
}
\end{lstlisting}
check if two segments' perp projection overlap
\begin{lstlisting}
inline bool seg_over_seg(const Point& a,const Point& b,
                         const Point& c,const Point& d){
  Vec u=b-a;
  Vec r0=vec_rotate_left(u),r1=vec_rotate_left(u);
  double t,k;
  if(ray_intersect(a,r0,c,d-c,&t,&k)) if(ls(k,1) && gr(k,0)) return true;
  if(ray_intersect(b,r1,c,d-c,&t,&k)) if(ls(k,1) && gr(k,0)) return true;
  Vec v=d-c;
  r0=vec_rotate_left(v);r1=vec_rotate_left(v);
  if(ray_intersect(c,r0,a,b-a,&t,&k)) if(ls(k,1) && gr(k,0)) return true;
  if(ray_intersect(d,r1,a,b-a,&t,&k)) if(ls(k,1) && gr(k,0)) return true;
  return false;
}
\end{lstlisting}
return two parallel segments' distance
\begin{lstlisting}
inline double dist_2seg(const Point& a,const Point& b,
                        const Point& c,const Point& d){
  return dist2line(c,a,b-a);
}
\end{lstlisting}
rotating calipers
\begin{lstlisting}
void rc(){
  int p,q;
  int sp;
  int cnt=0;
  double f;
  f=10001;
  for(int i=0;i<M;++i) if(P[i].x<f){f=P[i].x;p=i;}
  f=-10001;
  for(int i=0;i<N;++i) if(Q[i].x>f){f=Q[i].x;q=i;}
  while(gr(area(P[p],P[nextp(p)],Q[nextq(q)]),area(P[p],P[nextp(p)],Q[q]))) q=nextq(q);
  sp=p;

  if(point_over_seg(Q[q],P[p],P[nextp(p)]))
    mindist=min(mindist,dist2line(Q[q],P[p],P[nextp(p)]-P[p]));
  else mindist=min(mindist,(P[p]-Q[q]).len());
  do{
    int np=p,nq=q;
    if(gr(area(P[p],P[nextp(p)],Q[nextq(q)]),area(P[p],P[nextp(p)],Q[q]))) nq=nextq(q);
    if(gr(area(Q[q],Q[nextq(q)],P[nextp(p)]),area(Q[q],Q[nextq(q)],P[p]))) np=nextp(p);
    if(nq==q && np!=p){
      if(point_over_seg(Q[q],P[p],P[nextp(p)]))
        mindist=min(mindist,dist2line(Q[q],P[p],P[nextp(p)]-P[p]));
      else mindist=min(mindist,(Q[q]-P[nextp(p)]).len());
      p=nextp(p);
    }
    else if(np==p && nq!=q){
      if(point_over_seg(P[p],Q[q],Q[nextq(q)]))
        mindist=min(mindist,dist2line(P[p],Q[q],Q[nextq(q)]-Q[q]));
      else mindist=min(mindist,(P[p]-Q[nextq(q)]).len());
      q=nextq(q);
    }
    else if(np==p && nq==q){
      if(seg_over_seg(P[p],P[nextp(p)],Q[q],Q[nextq(q)]))
        mindist=min(mindist,dist_2seg(P[p],P[nextp(p)],Q[q],Q[nextq(q)]));
      else{
        //mindist=min(mindist,(P[nextp(p)]-Q[nextq(q)]).len());
        mindist=min(mindist,(P[nextp(p)]-Q[q]).len());
        //mindist=min(mindist,(Q[nextq(q)]-P[p]).len());
      }
      p=nextp(p);
    }
    if(p==sp) ++cnt;
  }while(!(cnt>1&&p==nextp(sp)));
}
\end{lstlisting}
\subsubsection{closed point pair}
O(NlogN) \\
sweep line \\ 
POJ 3714; ZOJ 2107 \\
compx, compy, compare functors used in sorted point sequence and BST respectively
\begin{lstlisting}
const int MAXN=200002;

Point P[MAXN];
int N;
struct compx{
  bool operator()(const Point& a,const Point& b)const{
    if(a.x==b.x) return ls(a.y,b.y);
    else return ls(a.x,b.x);
  }
};
struct compy{
  bool operator()(const Point& a,const Point& b)const{
    if(a.y==b.y) return ls(a.x,b.x);
    else return ls(a.y,b.y);
  }
};
\end{lstlisting}
sweep line algorithm
\begin{lstlisting}
double mindist;

Point eventp;
typedef set<Point,compy> Box;
typedef Box::iterator BoxIter;
void sweepline(){
  sort(P,P+2*N,compx());
  mindist=INF;
  int l=0;
  Box box;
  for(int i=0;i<2*N;++i){
    eventp=P[i];
    double x=eventp.x-mindist,negy=eventp.y-mindist,posy=eventp.y+mindist;

    while(l<i && ls(P[l].x,x)){
      box.erase(P[l]);
      ++l;
    }
    
    BoxIter a=box.lower_bound(Point(x,negy));
    BoxIter b=box.lower_bound(Point(x,posy));
    for(BoxIter iter=a;iter!=b;iter++){
      Vec d=*iter - eventp;
      if(iter->color != eventp.color) mindist=min(mindist,d.len());
    }
    box.insert(eventp);
  }
}
\end{lstlisting}
\subsubsection{closed circle pair}
O(NlogN) \\ 
sweep line  \\
HDU 3124  \\
data structure and compare functors
\begin{lstlisting}
struct Circle{
  Point c;double r;
  Circle(const Point& a=Point(),const double& b=0):c(a),r(b){
  }
};
const int MAXN=50000;
Circle C[MAXN];
int N;
Point P[MAXN];

struct compx{
  bool operator()(const Point& a,const Point& b)const{
    if(equ(a.x,b.x)) return ls(a.y,b.y);
    else return ls(a.x,b.x);
  }
};
struct compy{
  bool operator()(const Point& a,const Point& b)const{
    if(equ(a.y,b.y)) return ls(a.x,b.x);
    else return ls(a.y,b.y);
  }
};
\end{lstlisting}
sweep line algorithm, use circle's top and bottom vertex as event
point
\begin{lstlisting}
double mindist;

Point eventp;
typedef set<Point,compy> Box;
typedef Box::iterator BoxIter;
void sweepline(){
  sort(P,P+N,compx());
  mindist=INF;
  int l=0;
  Box box;
  for(int i=0;i<N;++i){
    eventp=P[i];
    Circle cir=C[eventp.c];
    double x=eventp.x-mindist,negy=cir.c.y-cir.r-mindist,posy=cir.c.y+cir.r+mindist;

    while(l<i && ls(C[P[l].c].c.x+C[P[l].c].r,x)){
      Circle tmp=C[P[l].c];
      box.erase(Point(tmp.c.x,tmp.c.y-tmp.r));
      box.erase(Point(tmp.c.x,tmp.c.y+tmp.r));
      ++l;
    }
    
    if(!box.empty()){
      BoxIter begin=box.lower_bound(Point(NEGINF,negy));
      BoxIter end=box.lower_bound(Point(NEGINF,posy));
      for(BoxIter iter=begin;iter!=end;++iter){
        Circle tmp=C[iter->c];
        double d=(tmp.c-cir.c).len() - tmp.r - cir.r;
        mindist=min(mindist,d);
      }
    }
    box.insert(Point(cir.c.x,cir.c.y+cir.r,eventp.c));
    box.insert(Point(cir.c.x,cir.c.y-cir.r,eventp.c));
  }
}
\end{lstlisting}
main function, save circle's top and bottom vertex
\begin{lstlisting}
int main(){
  int T;scanf("%d",&T);
  while(T-->0){
    scanf("%d",&N);
    for(int i=0;i<N;++i){
      int x,y,r;
      scanf("%d%d%d",&x,&y,&r);
      C[i].c.x=x;C[i].c.y=y;C[i].r=r;
      P[i]=Point(C[i].c.x-C[i].r,C[i].c.y,i);
    }
    sweepline();
    printf("%.6f\n",mindist);
  }
  return 0;
}
\end{lstlisting}
\subsubsection{circle hierarchy}
O(NlogN) \\
sweep line \\ 
event point is the most left and right vertex of a circle  \\
implicit store the interval cuted by the current sweep line. \\
POJ 2932 
data structures and compare functors \\
\begin{lstlisting}
const int MAXC=40000;
const int MAXN=2*MAXC;
struct Circle{
  Point c;double r;
  Circle(const Point& a=Point(),const double& b=0):c(a),r(b){}
};
//a extra INF radius circle
Circle circle[MAXC+1];
Point P[MAXN];
int N;
Point eventp;

struct compx{
  bool operator()(const Point& a,const Point& b)const{
    if(equ(a.x,b.x)) return ls(a.y,b.y);else return ls(a.x,b.x);
  }
};

struct Interval{
  int p;
  int i,j;
  bool lowi,lowj;
  Interval(int par=0,int a=0,bool a0=false,int b=0,bool b0=false):
    p(par),i(a),j(b),lowi(a0),lowj(b0){}
};
inline pair<double,double> get_itv(const Circle& c,const double& x){
  double d=fabs(c.c.x-x);
  double del=c.r*c.r - d*d;
  if(lseq(del,0.0f)) del=0.0f;
  del=sqrt(del);
  return make_pair(c.c.y-del,c.c.y+del);
}
struct compcircle{
  bool operator()(const Interval& a,const Interval& b)const{
    pair<double,double> itv0=get_itv(circle[a.i],eventp.x);
    pair<double,double> itv1=get_itv(circle[a.j],eventp.x);
    double x0,x1;
    if(a.lowi) x0=itv0.first;else x0=itv0.second;
    if(a.lowj) x1=itv1.first;else x1=itv1.second;

    itv0=get_itv(circle[b.i],eventp.x);
    itv1=get_itv(circle[b.j],eventp.x);
    double y0,y1;
    if(b.lowi) y0=itv0.first;else y0=itv0.second;
    if(b.lowj) y1=itv1.first;else y1=itv1.second;

    if(equ(x0,y0)) return ls(x1,y1);
    else return ls(x0,y0);
  }
};

//-----------union set------------
int parent[MAXC+1];
inline void make_set(){for(int i=0;i<=N;++i) parent[i]=i;}
//attach y to x
inline void link(int x,int y){parent[y]=x;}
\end{lstlisting}
sweep line \\
left and right point as event point \\
store following in the BST:(Interval structure)  \\
 -- parent \\
 -- which circle contribute the two end points \\
 -- if these end points is bottom vertex of the interval

\begin{lstlisting}
typedef set<Interval,compcircle> Itvl;
typedef Itvl::iterator Iter;
void debug(Itvl itvl){
  for(Iter k=itvl.begin();k!=itvl.end();++k){
    printf("parent: %d lower: %d %d  upper: %d %d\n",k->p,k->i,k->lowi,k->j,k->lowj);
  }
}
void sweepline(){
  make_set();
  sort(P,P+2*N,compx());
  Itvl itvl;
  itvl.insert(Interval(0,0,true,0,false));

  for(int i=0;i<2*N;++i){
    eventp=P[i];

    if(eventp.in){
      Iter lower=itvl.lower_bound(Interval(0,eventp.i,true,eventp.i,false));
      --lower;
      link(lower->p,eventp.i);
      Interval tmp=*lower;
      itvl.erase(lower);
      itvl.insert(Interval(tmp.p,tmp.i,tmp.lowi,eventp.i,true));
      itvl.insert(Interval(eventp.i,eventp.i,true,eventp.i,false));
      itvl.insert(Interval(tmp.p,eventp.i,false,tmp.j,tmp.lowj));
    }
    else{
      Iter d=itvl.find(Interval(0,eventp.i,true,eventp.i,false));
      Iter l=d;--l;
      Iter u=d;++u;
      Interval low=*l,up=*u;
      itvl.erase(d);itvl.erase(l);itvl.erase(u);
      itvl.insert(Interval(low.p,low.i,low.lowi,up.j,up.lowj));
    }
  }
}
\end{lstlisting}
as for POJ 2932 don't ask for the total hierarchy, so this is a
special sweep line algorithm to speed up.
\begin{lstlisting}
void sweepline(){
  make_set();
  sort(P,P+2*N,compx());
  Itvl itvl;
  itvl.insert(Interval(0,0,true,0,false));

  for(int i=0;i<2*N;++i){
    eventp=P[i];

    if(eventp.in){
      Iter lower=itvl.lower_bound(Interval(0,eventp.i,true,eventp.i,false));
      --lower;
      link(lower->p,eventp.i);
      if(lower->p == 0){
        Interval tmp=*lower;
        itvl.erase(lower);
        itvl.insert(Interval(tmp.p,tmp.i,tmp.lowi,eventp.i,true));
        itvl.insert(Interval(eventp.i,eventp.i,true,eventp.i,false));
        itvl.insert(Interval(tmp.p,eventp.i,false,tmp.j,tmp.lowj));
      }
    }
    else{
      Iter d=itvl.find(Interval(0,eventp.i,true,eventp.i,false));
      if(d==itvl.end()) continue;
      Iter l=d;--l;
      Iter u=d;++u;
      Interval low=*l,up=*u;
      itvl.erase(d);itvl.erase(l);itvl.erase(u);
      itvl.insert(Interval(low.p,low.i,low.lowi,up.j,up.lowj));
    }
  }
}
\end{lstlisting}
\end{document}  
